\documentclass{ctexbeamer}

\usetheme{Madrid}
\beamertemplatenavigationsymbolsempty

%\usecolortheme{beaver}
% the following colors are modified from beaver
\definecolor{tuna}{rgb}{0.098,0.51,0.996}
\definecolor{thu}{rgb}{0.50,0.36,0.71}

\setbeamercolor{section in toc}{fg=black,bg=white}
\setbeamercolor{item}{fg=tuna,bg=white}
\setbeamercolor{alerted text}{fg=tuna!80!gray}
\setbeamercolor*{palette primary}{fg=tuna!60!black,bg=gray!30!white}
\setbeamercolor*{palette secondary}{fg=tuna!70!black,bg=gray!15!white}
\setbeamercolor*{palette tertiary}{bg=tuna!80!black,fg=gray!10!white}
\setbeamercolor*{palette quaternary}{fg=tuna,bg=gray!5!white}

\setbeamercolor*{sidebar}{fg=tuna,bg=gray!15!white}

\setbeamercolor*{palette sidebar primary}{fg=tuna!10!black}
\setbeamercolor*{palette sidebar secondary}{fg=white}
\setbeamercolor*{palette sidebar tertiary}{fg=tuna!50!black}
\setbeamercolor*{palette sidebar quaternary}{fg=gray!10!white}

%\setbeamercolor*{titlelike}{parent=palette primary}
\setbeamercolor{titlelike}{parent=palette primary,fg=tuna}
\setbeamercolor{frametitle}{bg=gray!10!white}
\setbeamercolor{frametitle right}{bg=gray!60!white}

\setbeamercolor*{separation line}{}
\setbeamercolor*{fine separation line}{}

\setbeamertemplate{sections/subsections in toc}[square]
\setbeamertemplate{items}[square]
 
\usepackage{hyperref}

\newcommand{\T}[1]{\texttt{#1}}

\title{Mutt配置:一种实践的部署方式}
\author[Zenithal(则尼梭)]{Zenithal(则尼梭)\newline\url{i@zenithal.me}}
\date{2020-11-21}

\begin{document}

\begin{frame}
\maketitle
\end{frame}

\section{关于我}
\begin{frame}{关于我}
  \begin{itemize}
    \item 关于我\begin{itemize}
        \item 你清普通同学
        \item \url{https://github.com/ZenithalHourlyRate}
        \item PGP 公钥:\url{https://blog.zenithal.me/key}
        \item 有一个私有频道%,\T{AAAAAFk2Y-K5-0vHbiI8wg} (不会在群里放直接可以点的链接的)
        \item 有一个公有频道,但是被群友阻止发出来了%,请在我的用户名后面加上频道
        \item 尝试 Mutt 尝试了三回,终于在最后一回用了一个周末把事情搞定了
      \end{itemize}
    \item 在金枪鱼中\begin{itemize}
        \item 潜水摸鱼两三年
        %\item 上周公费旅游(别的人没有空罢了),负责发贴纸
        %\item 最近可能出现在一些邮件中
    \end{itemize}
  \end{itemize}
\end{frame}

\section{为什么选用 Mutt}
\begin{frame}{为什么选用 Mutt}
  \begin{itemize}
    \item 天下苦邮件旧矣\begin{itemize}
        \item 某猫40GiB邮件\qquad\includegraphics[height=3\fontcharht\font`\B]{felix.jpg}
        \item Webmail 功能不够用啊!
        \item 邮件客户端卡死了/漏内存啊!
        \item 各种诡异的需求满足不了啊!
      \end{itemize}
    \item 所有的邮件客户端坏坏,这个只是坏得少一些 --- Mutt 作者\footnote{
        \textit{All mail clients suck. This one just sucks less.} - Author of Mutt, circa 1995}
    \item 如果您很难向我发送加密的消息,请使用 Mutt --- 埃里克·雷蒙
      \footnote{\textit{If your mailer makes this(use GPG encryption when sending me messages) difficult, consider switching to Mutt.} - Eric. S. Raymond}
    \item $\backslash$壳/$\backslash$管道/$\backslash$退/
      \footnote{$\backslash$SHELL/$\backslash$PIPE/$\backslash$TUI/}
    \item 不要贵!不要老鼠!
      \footnote{GUI and mouse}
    \item 脚本!一行解决!(特指CLI)
  \end{itemize}
\end{frame}

\begin{frame}{Mutt 的高亮}
  \begin{itemize}
    \item Mutt 做了什么?它只是一个邮件客户端\begin{itemize}
        \item 读取邮件
        \item 构造邮件
    \end{itemize}
    \item 读取邮件\begin{itemize}
        \item 本地 \T{Maildir/mbox}\begin{itemize}
          \item 于是可以 \T{procmail} 进行奇怪处理
          \item 也可以在文件系统上做坏事,比如 FUSE 与 s3
        \end{itemize}
        \item \T{IMAP/POP3}(最近加入了\T{XOAUTH2})
        \item 正则搜索(最近加入了搜索模式提醒,确实太难背了)
        \item 解密与验证签名
        \item 邮件头/邮件体缓存
        \item 按照线程排序
    \end{itemize}
    \item 构造邮件\begin{itemize}
      \item 自定义邮件头
      \item 控制MIME的一些细节(比如发HTML邮件需要控制 Content-Type)
      \item 调用外部编辑器(比 webmail 那个 textarea 高到不知道哪里去了)
      \item 签名与加密
    \end{itemize}
  \end{itemize}
\end{frame}

\section{激动人心的展示环节!}
\begin{frame}{一些基本概念}
  \begin{itemize}
    \item 茵蒂克丝(index)窗口\begin{itemize}
        \item 一进入 Mutt 就看到的,展示所有邮件的简略信息
        \item 简略信息的格式可以调节
      \end{itemize}
    \item 页面(pager)窗口\begin{itemize}
        \item 用来浏览邮件内容
        \item 可以根据 mailcap 对邮件内容进行浏览(渲染html与pdf无压力!)
      \end{itemize}
    \item 写作(compose)窗口\begin{itemize}
        \item 对邮件的元信息进行调节
        \item 其余写作任务不是 Mutt 管的
      \end{itemize}
  \end{itemize}
\end{frame}

\begin{frame}{我的架构}
  \begin{itemize}
    \item 用 getmail6 收邮件到 Maildir 中
    \item 用 Mutt 读取该 Maildir,做出一些包括归档的操作
    \item 用 msmtp 发邮件,部分帐号配置了 DKIM 
    \item 其他同学用 Mutt 内嵌的 IMAP/SMTP 支持也可,不过不在讨论范围内
  \end{itemize}
\end{frame}

\begin{frame}{账户相关配置}
  \begin{itemize}
    \item 将 getmail6 与 msmtp 配置好\begin{itemize}
      \item QQ邮箱的独立密码
      \item Gmail的「允许较不安全应用」以及独立密码
      \item StartTLS
    \end{itemize}
    \item 或者不配置,先下载样例 Maildir 从\url{https://blog.zenithal.me/mail.tgz}
    \item 账户相关的配置放在 \T{\~{}/.mutt/muttrc.d/} 中
    \item 设置文件夹
    \item 设置名字与邮箱
    \item 设置签名
    \item 其余账户相关配置
  \end{itemize}
\end{frame}

\begin{frame}{账户无关配置}
  \begin{itemize}
    \item 账户无关的配置放在 \T{\~{}/.mutt/muttrc} 中
    \item 邮件头看哪些?
    \item 奇妙奇妙的选项,请参照 Manual\begin{itemize}
        \item 按线程排序
        \item 可以去\url{https://github.com/ZenithalHourlyRate/muttrc}获取一份我用的配置
    \end{itemize}
    \item 颜色,按键绑定,别名
  \end{itemize}
\end{frame}

\section{需求们}
\begin{frame}{多账户相关配置}
  \begin{itemize}
    \item folder-hook
    \item account-hook for IMAP user
    \item sidebar 
  \end{itemize}
\end{frame}

\begin{frame}{Mailcap}
  \begin{itemize}
    \item 上古神器!
    \item 扩展名太多打开不过来?
    \item autoview
    \item PDF/WORD
    \item 压缩文件
  \end{itemize}
\end{frame}

\begin{frame}{GnuPG}
  \begin{itemize}
    \item 中古神器!
    \item gpg-agent 是重头戏
    \item 直接copy paste Mutt提供的gpg.rc,配一下signing key和默认gpg选项。
  \end{itemize}
\end{frame}


\end{document}

% vim: nospell
